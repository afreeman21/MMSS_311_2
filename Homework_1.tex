\documentclass[]{article}
\usepackage{lmodern}
\usepackage{amssymb,amsmath}
\usepackage{ifxetex,ifluatex}
\usepackage{fixltx2e} % provides \textsubscript
\ifnum 0\ifxetex 1\fi\ifluatex 1\fi=0 % if pdftex
  \usepackage[T1]{fontenc}
  \usepackage[utf8]{inputenc}
\else % if luatex or xelatex
  \ifxetex
    \usepackage{mathspec}
  \else
    \usepackage{fontspec}
  \fi
  \defaultfontfeatures{Ligatures=TeX,Scale=MatchLowercase}
\fi
% use upquote if available, for straight quotes in verbatim environments
\IfFileExists{upquote.sty}{\usepackage{upquote}}{}
% use microtype if available
\IfFileExists{microtype.sty}{%
\usepackage{microtype}
\UseMicrotypeSet[protrusion]{basicmath} % disable protrusion for tt fonts
}{}
\usepackage[margin=1in]{geometry}
\usepackage{hyperref}
\hypersetup{unicode=true,
            pdftitle={Homework 1},
            pdfauthor={Alex Freeman},
            pdfborder={0 0 0},
            breaklinks=true}
\urlstyle{same}  % don't use monospace font for urls
\usepackage{color}
\usepackage{fancyvrb}
\newcommand{\VerbBar}{|}
\newcommand{\VERB}{\Verb[commandchars=\\\{\}]}
\DefineVerbatimEnvironment{Highlighting}{Verbatim}{commandchars=\\\{\}}
% Add ',fontsize=\small' for more characters per line
\usepackage{framed}
\definecolor{shadecolor}{RGB}{248,248,248}
\newenvironment{Shaded}{\begin{snugshade}}{\end{snugshade}}
\newcommand{\KeywordTok}[1]{\textcolor[rgb]{0.13,0.29,0.53}{\textbf{#1}}}
\newcommand{\DataTypeTok}[1]{\textcolor[rgb]{0.13,0.29,0.53}{#1}}
\newcommand{\DecValTok}[1]{\textcolor[rgb]{0.00,0.00,0.81}{#1}}
\newcommand{\BaseNTok}[1]{\textcolor[rgb]{0.00,0.00,0.81}{#1}}
\newcommand{\FloatTok}[1]{\textcolor[rgb]{0.00,0.00,0.81}{#1}}
\newcommand{\ConstantTok}[1]{\textcolor[rgb]{0.00,0.00,0.00}{#1}}
\newcommand{\CharTok}[1]{\textcolor[rgb]{0.31,0.60,0.02}{#1}}
\newcommand{\SpecialCharTok}[1]{\textcolor[rgb]{0.00,0.00,0.00}{#1}}
\newcommand{\StringTok}[1]{\textcolor[rgb]{0.31,0.60,0.02}{#1}}
\newcommand{\VerbatimStringTok}[1]{\textcolor[rgb]{0.31,0.60,0.02}{#1}}
\newcommand{\SpecialStringTok}[1]{\textcolor[rgb]{0.31,0.60,0.02}{#1}}
\newcommand{\ImportTok}[1]{#1}
\newcommand{\CommentTok}[1]{\textcolor[rgb]{0.56,0.35,0.01}{\textit{#1}}}
\newcommand{\DocumentationTok}[1]{\textcolor[rgb]{0.56,0.35,0.01}{\textbf{\textit{#1}}}}
\newcommand{\AnnotationTok}[1]{\textcolor[rgb]{0.56,0.35,0.01}{\textbf{\textit{#1}}}}
\newcommand{\CommentVarTok}[1]{\textcolor[rgb]{0.56,0.35,0.01}{\textbf{\textit{#1}}}}
\newcommand{\OtherTok}[1]{\textcolor[rgb]{0.56,0.35,0.01}{#1}}
\newcommand{\FunctionTok}[1]{\textcolor[rgb]{0.00,0.00,0.00}{#1}}
\newcommand{\VariableTok}[1]{\textcolor[rgb]{0.00,0.00,0.00}{#1}}
\newcommand{\ControlFlowTok}[1]{\textcolor[rgb]{0.13,0.29,0.53}{\textbf{#1}}}
\newcommand{\OperatorTok}[1]{\textcolor[rgb]{0.81,0.36,0.00}{\textbf{#1}}}
\newcommand{\BuiltInTok}[1]{#1}
\newcommand{\ExtensionTok}[1]{#1}
\newcommand{\PreprocessorTok}[1]{\textcolor[rgb]{0.56,0.35,0.01}{\textit{#1}}}
\newcommand{\AttributeTok}[1]{\textcolor[rgb]{0.77,0.63,0.00}{#1}}
\newcommand{\RegionMarkerTok}[1]{#1}
\newcommand{\InformationTok}[1]{\textcolor[rgb]{0.56,0.35,0.01}{\textbf{\textit{#1}}}}
\newcommand{\WarningTok}[1]{\textcolor[rgb]{0.56,0.35,0.01}{\textbf{\textit{#1}}}}
\newcommand{\AlertTok}[1]{\textcolor[rgb]{0.94,0.16,0.16}{#1}}
\newcommand{\ErrorTok}[1]{\textcolor[rgb]{0.64,0.00,0.00}{\textbf{#1}}}
\newcommand{\NormalTok}[1]{#1}
\usepackage{graphicx,grffile}
\makeatletter
\def\maxwidth{\ifdim\Gin@nat@width>\linewidth\linewidth\else\Gin@nat@width\fi}
\def\maxheight{\ifdim\Gin@nat@height>\textheight\textheight\else\Gin@nat@height\fi}
\makeatother
% Scale images if necessary, so that they will not overflow the page
% margins by default, and it is still possible to overwrite the defaults
% using explicit options in \includegraphics[width, height, ...]{}
\setkeys{Gin}{width=\maxwidth,height=\maxheight,keepaspectratio}
\IfFileExists{parskip.sty}{%
\usepackage{parskip}
}{% else
\setlength{\parindent}{0pt}
\setlength{\parskip}{6pt plus 2pt minus 1pt}
}
\setlength{\emergencystretch}{3em}  % prevent overfull lines
\providecommand{\tightlist}{%
  \setlength{\itemsep}{0pt}\setlength{\parskip}{0pt}}
\setcounter{secnumdepth}{0}
% Redefines (sub)paragraphs to behave more like sections
\ifx\paragraph\undefined\else
\let\oldparagraph\paragraph
\renewcommand{\paragraph}[1]{\oldparagraph{#1}\mbox{}}
\fi
\ifx\subparagraph\undefined\else
\let\oldsubparagraph\subparagraph
\renewcommand{\subparagraph}[1]{\oldsubparagraph{#1}\mbox{}}
\fi

%%% Use protect on footnotes to avoid problems with footnotes in titles
\let\rmarkdownfootnote\footnote%
\def\footnote{\protect\rmarkdownfootnote}

%%% Change title format to be more compact
\usepackage{titling}

% Create subtitle command for use in maketitle
\providecommand{\subtitle}[1]{
  \posttitle{
    \begin{center}\large#1\end{center}
    }
}

\setlength{\droptitle}{-2em}

  \title{Homework 1}
    \pretitle{\vspace{\droptitle}\centering\huge}
  \posttitle{\par}
    \author{Alex Freeman}
    \preauthor{\centering\large\emph}
  \postauthor{\par}
      \predate{\centering\large\emph}
  \postdate{\par}
    \date{4/19/2019}


\begin{document}
\maketitle

\subsection{R Markdown}\label{r-markdown}

This is an R Markdown document. Markdown is a simple formatting syntax
for authoring HTML, PDF, and MS Word documents. For more details on
using R Markdown see \url{http://rmarkdown.rstudio.com}.

When you click the \textbf{Knit} button a document will be generated
that includes both content as well as the output of any embedded R code
chunks within the document. You can embed an R code chunk like this:

\begin{Shaded}
\begin{Highlighting}[]
\KeywordTok{summary}\NormalTok{(cars)}
\end{Highlighting}
\end{Shaded}

\begin{verbatim}
##      speed           dist       
##  Min.   : 4.0   Min.   :  2.00  
##  1st Qu.:12.0   1st Qu.: 26.00  
##  Median :15.0   Median : 36.00  
##  Mean   :15.4   Mean   : 42.98  
##  3rd Qu.:19.0   3rd Qu.: 56.00  
##  Max.   :25.0   Max.   :120.00
\end{verbatim}

\subsection{Including Plots}\label{including-plots}

You can also embed plots, for example:

\includegraphics{Homework_1_files/figure-latex/pressure-1.pdf}

Note that the \texttt{echo\ =\ FALSE} parameter was added to the code
chunk to prevent printing of the R code that generated the plot.

library(dplyr) library(ggplot2) library(broom)

\section{Regression}\label{regression}

\section{a}\label{a}

sick\_data\(sickornot <- ifelse(sick_data\)result == ``Positive'', 1, 0)
sick\_ols \textless{}- lm (sickornot \textasciitilde{} temp + bp, data =
sick\_data) \#Intercept: -5.213456, temp coef = 0.062819, bp coef =
-0.008287

sick\_logit \textless{}- glm(sickornot \textasciitilde{} temp + bp, data
= sick\_data, family=binomial(link=``logit'')) \#Intercept: -199.327,
temp coef = 2.314, bp coef = -0.35

\section{b}\label{b}

predictionlogit \textless{}- plogis(predict(sick\_logit, sick\_data))
predictionOLS \textless{}- predict(sick\_ols, sick\_data)
summary(predictionlogit) summary(predictionOLS)

\section{accuracy of OLS}\label{accuracy-of-ols}

sick\_data resultOLS \textless{}- ifelse(predictionOLS \textgreater{}=
.5, 1, 0) correctpredictions \textless{}- ifelse(sick\_data\$sickornot
== resultOLS, 1, 0) print(correctpredictions)
sum(correctpredictions)/1000 \#The OLS regression is 96.4\% accurate

\section{accuracy of Logit}\label{accuracy-of-logit}

sick\_data resultLogit \textless{}- ifelse(predictionlogit
\textgreater{}= .5, 1, 0) correctpredictions \textless{}-
ifelse(sick\_data\$sickornot == resultLogit, 1, 0)
print(correctpredictions) sum(correctpredictions)/1000 \#The Logit
regression is 99.2\% accurate

\section{c}\label{c}

\section{temperature in terms of blood pressure (OLS) when y-hat =
0.5}\label{temperature-in-terms-of-blood-pressure-ols-when-y-hat-0.5}

\section{Let y = B1(bp) + B2 (temp) + B0}\label{let-y-b1bp-b2-temp-b0}

\section{Then for OLS we have 0.5 = -0.0082865(bp) + 0.0628185(temp) -
5.2134563}\label{then-for-ols-we-have-0.5--0.0082865bp-0.0628185temp---5.2134563}

\section{So, temp = 90.95181 +
0.1319118(bp)}\label{so-temp-90.95181-0.1319118bp}

\section{For logit, we know y = (e\^{}(B1(bp) + B2 (temp) + B0)) / (1+
e\^{}(B1(bp) + B2 (temp) +
B0))}\label{for-logit-we-know-y-eb1bp-b2-temp-b0-1-eb1bp-b2-temp-b0}

\section{Thus, we have 0.5 = e\^{}(-0.3499(bp) + 2.3140(temp) -199.3267)
/ (1 + e\^{}(-0.3499(bp) + 2.3140(temp)
-199.3267))}\label{thus-we-have-0.5-e-0.3499bp-2.3140temp--199.3267-1-e-0.3499bp-2.3140temp--199.3267}

\section{Thus, temp = 0.432152 ln( (3.68541)(10\^{}86) *
2.71828\^{}(0.3499(bp))}\label{thus-temp-0.432152-ln-3.685411086-2.718280.3499bp}

\section{d}\label{d}

install.packages(``ggplot2'') library(ggplot2) ggplot(sick\_data,
aes(x=bp, y=temp, color=result)) +geom\_point() +geom\_abline(intercept
= 90.95181, slope = 0.1319118) + stat\_function(fun = function(x)
0.432152 * log((3.68541)\emph{(10\^{}86) } 2.71828\^{}(0.3499*(x))),
color = ``blue'')

\section{The blue line is the logit equation and the black line is the
OLS
equation}\label{the-blue-line-is-the-logit-equation-and-the-black-line-is-the-ols-equation}

\section{Question 2}\label{question-2}

\section{a}\label{a-1}

library(readr) widget\_data \textless{}-
read.csv(``Downloads/widget\_data.csv'') plot(widget\_data\$y)

\section{b}\label{b-1}

install.packages(``glmnet'') library(glmnet) ridge\_regression
\textless{}- glmnet(x= as.matrix(widget\_data{[}, -1{]}), y =
widget\_data\$y, alpha=0, lambda = .01:100) print(ridge\_regression)

\section{c}\label{c-1}

install.packages(``broom'') library(broom) tidy(ridge\_regression)
ridge\_data \textless{}- as.data.frame(tidy(ridge\_regression))
ggplot(data = ridge\_data, aes(x =
ridge\_data\(lambda, y = ridge_data\)estimate)) + geom\_line()

\section{d}\label{d-1}

install.packages(``glmnet'') library(glmnet) cv.glmnet(x =
as.matrix(widget\_data{[}, -1{]}), y =
widget\_data\(y, alpha=0, lambda = .01:100) glmnet <- cv.glmnet(x = as.matrix(widget_data[, -1]), y = widget_data\)y,
alpha=0, lambda = .01:100) summary(glmnet)

\section{e}\label{e}

lasso\_regression \textless{}- glmnet(x= as.matrix(widget\_data{[},
-1{]}), y = widget\_data\$y, alpha=1, lambda = .01:100)
print(lasso\_regression)

install.packages(``broom'') library(broom) install.packages(``ggplot2'')
library(ggplot2) tidy(lasso\_regression) print(lasso\_regression)
lasso\_data \textless{}- as.data.frame(tidy(lasso\_regression))
ggplot(data = lasso\_data, aes(x =
lasso\_data\(lambda, y = lasso_data\)estimate)) + geom\_line()

library(glmnet) cv.glmnet(x = as.matrix(widget\_data{[}, -1{]}), y =
widget\_data\(y, alpha=1, lambda = .01:100) glmnet_lasso <- cv.glmnet(x = as.matrix(widget_data[, -1]), y = widget_data\)y,
alpha=1, lambda = .01:100) summary(glmnet) \#f The lasso regression
gives a more complete estimation of the covariates. While the ridge
regression lowers the variance in estimation, it does not provide as
concrete of an answer

\section{Question 3}\label{question-3}

install.packages(``e1071'') library(e1071) install.packages(``caret'')
library(caret)

\section{a}\label{a-2}

library(readr) poll\_data \textless{}-
read.csv(``Downloads/pol\_data.csv'') View(poll\_data)

\section{b}\label{b-2}

set.seed(1) split \textless{}- 2/3 training\_data \textless{}-
createDataPartition(poll\_data\$group, p=split, list=F) train
\textless{}- poll\_data{[}training\_data, {]} test \textless{}-
poll\_data{[}-training\_data, {]}

\section{c}\label{c-2}

NB \textless{}- naiveBayes(group \textasciitilde{} pol\_margin +
col\_degree + house\_income, data=poll\_data, laplace = 0, train)
print(NB) SVC \textless{}- svm(train\$group \textasciitilde{} . ,
data=train) print(SVC)

predNB \textless{}-predict(NB, test) summary(predNB) predSVC
\textless{}- predict(SVC, test) summary(predSVC)

\section{d}\label{d-2}

print(test) print(predNB) correctpredictionsNB \textless{}-
ifelse(test\$group == predNB, 1, 0) sum(correctpredictionsNB)
print(correctpredictionsNB)/100

correctpredictionsSVC \textless{}- ifelse(test\$group == predSVC, 1, 0)
sum(correctpredictionsSVC) print(correctpredictionsSVC)/100

table(test\(group, predNB) table(test\)group, predSVC)


\end{document}
